\documentclass{article}
\usepackage{graphicx} 
\usepackage{fullpage}
\usepackage{amssymb,amsmath,amsthm,mathtools}


\newtheorem{theorem}{Theorem}
\newtheorem{thm}[theorem]{Theorem}
\newtheorem{lemma}[theorem]{Lemma}
\newtheorem{cor}[theorem]{Corollary}
\newtheorem{obs}[theorem]{Observation}
\newtheorem{conj}[theorem]{Conjecture}
\newtheorem{dfn}[theorem]{Definition}
\newtheorem{prob}[theorem]{Problem}
\newtheorem{claim}[theorem]{Claim}


\title{Generalized Kendall $\tau$}
\author{ }
\date{June 2025}

\begin{document}

\maketitle

\section*{Introduction}
Generalized Kendall Tau for partial rankings[1]

\section*{Methods}

The sorting problem (given a permutation $\pi$, find min number of gen.adjacent translocations to transform $\pi$ to 1,2,..,n) is NP-hard. 
It is interesting to find permutations with max distance to 1,2,..,n.
Or in general, find distances to 1,2,..,n for all permutations of [1,n]. 

Algorithm:

Make a list of all permutations of $\{1,2,..,n\}$ and set distance $D[\pi]$ as 0 for $\pi=(1,2,..,n)$ and a large int $X$ for other permutations $\pi$.

for dist=0 by 1 to infty\\
for each permutation $\pi$\\
if $D[\pi]=dist$ try every possible adjacent translocation on $\pi$ making a new permutation $\sigma$.\\
If $D[\sigma]=X$ set $D[\sigma]=dist+1$.\\
If $\pi$ is not found for dist then stop

\section*{Glossary}
\textbf{Translocation:} swapping two consecutive sequences. \\
\textbf{Translocation distance:} is the minimum number of operation that are needed to transform one permutation into another. \\
\textbf{Partial Ranking:} 

\vspace{0.5cm}

% ---------------------------------
\section{Results}

Let $d(\pi,\sigma)$ be the translocation distance between permutations $\pi $ and $\sigma $.

\begin{theorem}
$d(\pi,\sigma)=n-1$ for $\pi=(1,2,\dots,n)$ and $\sigma=(n,n-1,\dots,1)$.
\end{theorem}

\begin{proof}
{\em Upper bound}. 
We prove by induction that $d(\pi,\sigma)\le n-1$. 

\textbf{Base case}. If $n=1$ then $\pi$ = (1) = $\sigma$ and no translocation needed.

\textbf{Inductive step}.
Suppose $d(\pi,\sigma)\le n-1$.
Let $\pi'=(1,2,\dots,n-1)$ and $\sigma'=(n-1,n-2,\dots,1)$.
We prove $d(\pi,\sigma)\le n-1$ assuming that $d(\pi',\sigma')\le n-2$.

We make a translocation of $n$ and $(1,2,...,n-1)$ in $\pi$ producing permutation $(n, 1, 2, \dots, n-1)$.
It is $\pi'$ after $n$ and we can apply $n-2$ translocations to transform it to $\sigma'$ by induction hypothesis. 
This results in $n-1$ translocations transforming $\pi$ to $\sigma$.

{\em Lower bound}. ...
\end{proof}

When n = 2, we have 1 distinct ways of using exact 2 adjacent interval translocation:

12 $->$ 21

When n = 3, we have 4 distinct ways of using exact 1 adjacent interval translocation:

123 $->$ 312 $->$ 321

123 $->$ 213 $->$ 321

123 $->$ 132 $->$ 321

123 $->$ 231 $->$ 321


\section*{Bibliography}
[1] Yoo Y, Escobedo AR. A new binary programming formulation and social
choice property for Kemeny rank aggregation. Decision Analysis 2021;18(4):
296–320.

\end{document}
